
\section{Conclusiones}

Se requería implementar un medio para simular batallas entre bandos distintos. Para ello, se implementó un simulador de batallas el cual es personalizable a través de un lenguaje de dominio específico Battle Script. La simulación de batallas en un entorno controlado ayudaría a reducir el costo en vidas humanas en las guerras, así como ahorrar recursos económicos y tomar decisiones estratégicas.

Se implementaron unidades que funcionan como agentes casi puramente reactivos, así como bandos que representan los ejércitos, batallones, compañías, escuadrones, etc. Las unidades funcionan a través de un sistema experto que les aporta la inteligencia necesaria para actuar con el fin de eliminar a todas las unidades del o de los bandos contrarios.  Se definió un módulo para la generación automática de mapas de alturas.

Se probó la simulación con distintos casos de prueba, con mapas con poco o mucho relieve, bandos grandes o pequeños, así como se modificó el comportamiento de nuevas unidades definidas a través del lenguaje. Cabe destacar, que la simulación se ejecuta de forma eficiente, de forma que el usuario final obtiene resultados en poco tiempo.

Se sugiere desarrollar las sugerencias o recomendaciones para que el entorno y las unidades se asemejen cada vez más a los soldados, maquinarias de guerra, o ejércitos reales.