\section{La Simulaci\'on}

Como planteamos en el cap\'itulo anterior, se quiere desarrollar un programa que permita la simulaci\'on de enfrentamientos b\'elicos entre dos o m\'as bandos.

Para esto se tienen pensado los siguientes aspectos que van a ser fijos en cada una de las simulaciones:

\begin{enumerate}
	\item La existencia de un mapa o terreno donde se producir\'a el enfrentamiento. Este tendr\'a propiedades que se ser\'an modificables como las dimensiones, el relieve, la hidrograf\'ia, etc. La idea es que este se represente por una matriz bidimensional.
	
	\item Las acciones ser\'an por turnos. Como tenemos dos bandos les llamaremos: bando A y bando B. En el turno del bando A cada una de las unidades de A realizar\'a una y solo una acci\'on (ya sea moverse hacia otra posici\'on, atacar o mantener la posici\'on). Luego de esto se pasar\'a al turno del bando B, que al igual que A, podr\'a hacer una y solo una acci\'on con cada una de sus unidades. Esta forma de implementaci\'on permite un comportamiento de acci\'on-reacci\'on entre los dos bandos, asemej\'andose a lo que ocurre en la vida real.
	
\end{enumerate}

\subsection{El mapa}

El mapa consiste en una matriz bidimensional de $m$ filas y $n$ columnas. Cada objeto de la matriz es un objeto \verb|Cell|. Un objeto \verb|Cell| representa cada una de las casillas que conforman el mapa y tiene los siguientes atributos:

\begin{itemize}
	\item \verb|passable|: Valor entre 0 y 10 que indica cuan accesible es una celda. Las unidades tienden a buscar las celdas que tengan este par\'ametro lo m\'as alto posible, pues mientras mayor sea este valor, pueden hacer ataques m\'as poderosos.
	\item \verb|row|: Este par\'ametro es un n\'umero entero que indica la fila en la que se encuentra la celda.
	\item \verb|col|: Este par\'ametro es un n\'umero entero que indica la columna en que se encuentra la celda
	\item \verb|height|: Este par\'ametro es un n\'umero entre 0 y 1 que indica la altura de la casilla. En dependencia de este par\'ametro la casilla ser\'a terrestre o marina.
	\item \verb|bs_object|: Este par\'ametro hace referencia al objeto que se encuentra en la casilla. Si en la casilla no se encuentra ning\'un objeto entonces este par\'ametro es \verb|None|.
\end{itemize}

El mapa para una simulaci\'on se crea instanciando una clase \verb|LandMap| con los siguientes argumentos:

\begin{itemize}
	\item N\'umero de filas $m$
	\item N\'umero de columnas $n$
	\item Un array bidimensional de $m$ filas por $n$ columnas tal que cada posici\'on $i,j$ del array es un n\'umero entre 0 y 10 que indica cuan accesible es la celda $i,j$
	\item Un array bidimensional de $m$ filas por $n$ columnas tal que cada posici\'on $i,j$ del array es un n\'umero entre 0 y 1 que indica la altura de celda $i,j$
	\item Un n\'umero entre 0 y 1 que indica el nivel del mar. Todas las celdas cuya altura sea menor a este n\'umero ser\'an consideradas como celdas marinas y todas las celdas superior a este n\'umero ser\'an consideradas terrestres.
\end{itemize}

\subsection{Los objetos}

Un objeto es todo lo que se puede poner el mapa y cada objeto ocupa una y solo una casilla del mapa. Estos tienen propiedades como el id que s \'unico para cada objeto, los puntos de vida que determinan el estado de un objeto y la defensa un par\'ametro que indica cuan resistente es un objeto a los da\~{n}os que puede sufrir durante la simulaci\'on. Todos estos par\'ametros son n\'umeros de 1 a 10. Cuando la vida de un objeto llegue a 0, este se destruye desapareciendo del mapa. Los objetos se clasifican en dos tipos: unidades y objetos est\'aticos. Los objetos adem\'as tienen definidas dos funciones \verb|put_in_cell| cuya funci\'on es colocar al objeto en alguna posici\'on de un mapa. Si se intenta poner un objeto en una posici\'on diferente a su tipo este autom\'aticamente se destruye. La otra funci\'on \verb|take_damage|, a partir de un ataque sufrido indica como se reducen los puntos de vida del objeto.

\subsubsection{Objetos est\'aticos}

Los objetos est\'aticos los podemos definir como objetos propios del ambiente. Estos no pertenecen a ning\'un bando y no pueden realizar acciones pero si pueden ser afectados por las acciones que realicen las unidades. Estos solo pueden ser puestos en celdas terrestres. Ejemplos de estos objetos pueden ser \'arboles, rocas, muros, etc. Todos estos objetos son terrestres. 

\subsubsection{Unidades}

Las unidades son los agentes de la simulaci\'on. El objetivo de cada una de las unidades es destruir a las unidades enemigas (las que no pertenecen a su mismo bando), y para ello podr\'a analizar parte del ambiente en el que se encuentra y tomar la decisi\'on que sea m\'as conveniente seg\'un sean las circunstancias. 

Dado que el ambiente estar\'a cambiando constantemente, las unidades ser\'an agentes casi puramente reactivos. La arquitectura empleada para definir el comportamiento de las unidades es la Arquitectura de Brooks (de categorización o inclusión) que recordemos tiene las siguientes caracter\'isticas: 

\begin{itemize}
	\item La toma de decisión se realiza a través de un conjunto de comportamientos para lograr objetivos (reglas de la forma situación $\rightarrow$ acción).
	\item Las reglas pueden dispararse de manera simultánea por lo que debe un mecanismo para escoger entre ellas.	
\end{itemize}

Dado esto se defini\'o un sistema experto que actuar\'a como la funci\'on del agente. Este ser\'a descrito posteriormente.

\textbf{Propiedades de las unidades}

Las unidades adem\'as de las descritas anteriormente que tienen todos los objetos cuentan con las siguientes propiedades:

\begin{itemize}
	\item \verb|side|: Una instancia de la clase \verb|Side| que indica el bando al que pertenece la unidad
	\item \verb|attack|: Valor entre 1 y 10 que marca la capacidad de causar da\~{n}os a sus oponentes.
	\item \verb|moral|: Valor entre 1 y 10 que marca la moral con la unidad encara la batalla. Cuanto mayor es ese valor m\'as efectivos son sus ataques y sus defensas.
	\item \verb|ofensive|: Valor entre 1 y 10 que indica cuan ofensiva es una unidad. Un valor alto la hace m\'as ofensiva y un valor bajo la hace m\'as defensiva.
	\item \verb|min_range|: Valor entero entre 0 y 10 que indica el rango m\'inimo al que se debe encontrar un enemigo para que la unidad pueda atacarlo.
	\item \verb|max_range|: Valor entero entre 0 y 10 que indica el rango m\'aximo al que se debe encontrar un enemigo para que la unidad pueda atacarlo.
	\item \verb|radio|: Valor entero entre 1 y 9 que indica el n\'umero de casillas que son afectadas por un ataque de la unidad. Si es 1 se afecta solo a la casilla seleccionada para el ataque. Si es 9 se afectan la casilla seleccionada y las 8 adyacentes a esta. Si es 1 $<$ \verb|radio| $<$ 9, entonces se toman como casillas afectadas la seleccionada y $\verb|radio|-1$ casillas adyacentes a esta.
	\item \verb|vision| : Valor entre 1 y 10 que indica la cantidad de celdas en una determinada direcci\'on, que la unidad puede ``ver'' (saber que objetos est\'an en dicha celda).   
	\item \verb|intelligence|: Valor entre 0 y 10 que indica la inteligencia de la unidad. Mientras m\'as inteligente sea una unidad con mayor precisi\'on puede calcular los atributos de sus enemigos.
	\item \verb|recharge_turns|: Turnos que demora la unidad en recargar despu\'es de hacer un ataque. Mientras est\'e recargando la unidad no podr\'a atacar pero si puede moverse.
	\item \verb|solidarity|: Valor booleano que indica si la unidad es solidaria o no.
	\item \verb|movil|: Valor booleano que indica si la unidad puede desplazarse por el mapa.  
\end{itemize}

\subsubsection{Sistema experto}

A continuaci\'on se explicar\'a el sistema experto implementado que act\'ua como funci\'on del agente la cual describe el comportamiento del agente durante un turno.


Lo primero que la funci\'on hace es buscar si se existe un enemigo al que se pueda atacar. Para eso primero se comprueba si la unidad no est\'a recargando, si lo est\'a se reduce en uno los turnos que debe esperar para atacar, si no lo est\'a se busca el mejor enemigo para atacar. 
 
El mejor enemigo para atacar se determina de la siguiente forma:
 
Por cada casilla atacable (casilla que se encuentra a una distancia de la unidad entre su rango m\'inimo y su rango m\'aximo), se chequea si en la casilla hay un enemigo. Si el radio de ataque es mayor que 1 y hay una unidad amiga cerca del enemigo que pudiera verse afectada por el ataque, se ignora este enemigo. Esto se defini\'o as\'i para evitar el da\~{n}o ocasionado por fuego amigo.
 
Entonces si el enemigo no es ignorado debido a lo anterior, se calcula el costo de atacar al enemigo. Para dicho c\'alculo la unidad estima la vida y la defensa del enemigo y con estas estimaciones calcula cuantos turnos podr\'ia tomarle destruir a dicho enemigo. Mientras mayor sea la inteligencia de la unidad m\'as precisas ser\'an las estimaciones.

Luego de haber analizado todos los enemigos, el seleccionado por la unidad para atacarlo es aquel cuyo costo es menor. Si no se detecta ning\'un posible enemigo a atacar entonces la unidad no realiza un ataque en el turno.
 
Si la unidad no realiza un ataque, entonces, en caso de que se m\'ovil, chequea si se puede mover a alguna casilla adyacente a la que se encuentra. Esto se hace de la siguiente forma:
 
Se fija un costo en infinito. Luego por cada una de las celdas que la rodean en todos los puntos cardinales posibles (NW, N, NE, W, E, SW, S y SE)ase calcula el costo de moverse a dicha celda, y nos quedamos con la celda cuyo co es el menor. Luego si el menor costo detectado es menor a infinito la unidad se mueve a dicha celda en caso contrario la unidad mantiene su posici\'on. 
 
Ahora veamos como calcular el costo de que una unidad se mueva de una celda a otra.
 
Si la celda nueva es intransitable (tiene el par\'ametro \verb|passable| en 0), el tipo de la celda es diferente al tipo de la unidad o si en la celda hay alg\'un objeto se devuelve infinito. Estos son todos los casos en los que la unidad no puede moverse a dicha celda. Para las unidades terrestres siola diferencia de alturas entre dosla ltas es muy grande (mayor a 0.3), tampoco pueden avanzar a dicha celda, retorn\'andose infinito. 
 
Entonces en un primer momento se fija el costo en 10 - \verb|passable|/2, de esta manera se premia ir a celdas m\'as transitables. A continuaci\'on se comprueba si esa celda est\'a entre las que la unidad recuerda como ya visitadas. Si as\'i es, el coste se incrementar\'a en \verb|passable|/3. As\'i logramos incentivar que las unidades visiten celdas no visitadas con anterioridad  
  
Ahora se comprueba si la celda se encuentra en zona ``amiga'', es decir, si esa celda es adyacente a alguna celda en la que se encuentre alg\'un compa\~{n}ero de su bando. Si as\'i es y nuestra unidad protagonista es solidaria, el coste se reducir\'a a la mitad. Si es una zona amiga pero la unidad no es solidaria, el coste solo se reducir\'a dividi\'endose por la ra\'iz cuadrada de 2. De esta manera se incentiva que las unidades tiendan a permanecer en grupo y m\'as cuanto m\'as solidarias son.

Si la celda que se est\'a estudiando es una celda en la que nuestra unidad tendr\'a al alcance un enemigo, su coste se reducir\'a en el valor del par\'ametro \verb|ofensive|  multiplicado por la inversa de la ra\'iz cuadrada de la distancia m\'inima hasta el enemigo. Adem\'as la unidad observa las celdas cercanas a la celda que se est\'a estudiando que est\'en dentro de su rango de visi\'on. Si detecta que al moverse a dicha celda se encontrar\'a en rango de alg\'un enemigo el costo se aumenta en 1.1 por cada enemigo que pudiera atacar a la unidad. De esta manera se est\'a diciendo que, cuanto m\'as ofensiva sea nuestra unidad, m\'as se aproximar\'a al enemigo, aunque con cierta precauci\'on.  