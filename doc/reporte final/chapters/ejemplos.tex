\section{Ejemplos}

A continuación se muestran algunos ejemplos de uso del compilador y de ejecución de la simulación.

El primer ejemplo que se define a continuación es la creación de la clase Soldier que es un LandUnit. En este caso solo se define el constructor de la clase, que recibe el parámetro \verb|id| que representa el identificador de la unidad en la simulación y el par\'ametro \verb|attack| que es el valor de ataque de la unidad.

\begin{verbatim}
class Soldier is LandUnit -> {  
    constructor(number id, number attack) -> {
        number self.id = id;
        number self.attack = attack;
    };
}; 
\end{verbatim}

Note que tras cualquier instrucción en el lenguaje Battle Script se necesita poner un punto y coma (;) para especificar el fin de la instrucción. En este caso solo se indican los par\'ametros \verb|id| y \verb|attack|. Por tanto el resto de par\'ametros de la unidad tomar\'a los valores por defecto. Dentro de una clase se pueden definir funciones como se muestra en el siguiente ejemplo:

\begin{verbatim}
class Archer is LandUnit -> {
    constructor(number id, number max_range) -> {
        number self.id = id;
        number self.max_range = max_range;
    };
		
    function number plus_id(Archer arch) -> {
        return arch.id + 1;
    };
};
\end{verbatim}

Una funci\'on, definida dentro de una clase, debe definirse con al menos un par\'ametro cuyo tipo debe coincidir con el tipo de la clase pues este hace referencia a la instancia de la clase.

Las definiciones de clases deben ir todas al inicio del programa y cuando se terminen definir todas se debe poner el caracter '\&' para indicar que ya se termin\'o la definici\'on de las clases y ahora se van a comenzar escribir las instrucciones.

Para generar un mapa aleatorio se hace de la siguiente forma:

\begin{verbatim}
	LandMap map = build_random_map(1, 5, 5);
\end{verbatim} 

La funci\'on build\_random\_map es una funci\'on built-in que recibe como par\'ametros un n\'umero real entre 0 y 1 que indica el porcentaje de casillas de tierra que se desea tenga el mapa, y dos enteros que indican el n\'umero de filas y de columnas que se desea tenga el mapa.

Entonces pasemos a ver como crear las unidades:

\begin{verbatim}
Soldier sOne = Soldier(1, 10);
Soldier sTwo = Soldier(2, 9);
	
Archer aOne = Archer(3, 5);
Archer aTwo = Archer(4, 5);	
\end{verbatim}  

En este ejemplo hemos creado dos unidades de tipo ``Soldier'' y dos unidades de tipo ``Archer'' utilizando las clases definidas anteriormente.

Para poner una unidad en el mapa se hace de la siguiente forma:

\begin{verbatim}
sOne.put_in_cell(map, 0, 0);
\end{verbatim}

Se toma a la unidad y se hace un llamado a la funci\'on \verb|put_in_cell| pas\'andole como argumentos el mapa y par de enteros que indican la fila y la columna de la celda donde se desea colocar la unidad.

Un bando se crean de la siguiente forma:
  
\begin{verbatim}
Side SOne = Side(1, [sOne, aTwo]);
\end{verbatim}

Como se muestra en los ejemplos anteriores un bando se instancia pas\'andole un id y la lista de unidades que conforman el bando.

Para crear un simulador se hace de la siguiente manera:

\begin{verbatim}
Simulator sim = Simulator(map, [SOne, STwo], 20, 1);
\end{verbatim}

F\'ijese que un simulador se instancia con un mapa, la lista de los bandos, la cantidad de turnos m\'aximos que se desea dure la simulaci\'on y un n\'umero que indica la duraci\'on de un turno.

Para echar andar la simulaci\'on lo hacemos de la siguiente forma: 

\begin{verbatim}
sim.start();
\end{verbatim}

Entonces un ejemplo completo de un programa podr\'ia ser el siguiente:

\begin{verbatim}
class Soldier is LandUnit -> {
    constructor(number id, number attack) -> {
        number self.id = id;
        number self.attack = attack;
    };
};

class Archer is LandUnit -> {
    constructor(number id, number max_range) -> {
        number self.id = id;
        number self.max_range = max_range;
    };
};
&
LandMap map = build_random_map(1, 5, 5);

Soldier sOne = Soldier(1, 10);
Soldier sTwo = Soldier(2, 9);

Archer aOne = Archer(3, 5);
Archer aTwo = Archer(4, 5);

sOne.put_in_cell(map, 0, 0);
aTwo.put_in_cell(map, 0, 1);

aOne.put_in_cell(map, 4,4 );
sTwo.put_in_cell(map, 4, 3);

Side SOne = Side(1, [sOne, aTwo]);
Side STwo = Side(2, [aOne, sTwo]);

Simulator sim = Simulator(map, [SOne, STwo], 20, 1);

sim.start();
	
\end{verbatim}

En el lenguaje adem\'as se pueden definir instrucciones m\'as complicadas tales como: if, while, declaraciones, asignaciones, etc. A continuaci\'on se muestran algunos ejemplos de como hacerlo:

\begin{verbatim}
function number W(number a) -> { 
    if a lt 0 and a eq 0 -> { 
        return -1 * a;  
    } 
    elif a gt 1 -> {
        return 4;
    } 
    else -> { 
        return a; 
    }; 
};

function number A(number a) -> { 
    while True -> {
        a = a + 1;
    }; 
    return a;
};
\end{verbatim} 


