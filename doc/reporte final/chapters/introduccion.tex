\section{Introducción}

A lo largo de la historia, los conflictos bélicos han estado fuertemente ligados al desarrollo de la humanidad. Ejemplo de esto es la ``caza de cabezas'' y otras prácticas de lucha endémica en las sociedades agrarias en la prehistoria que se disputaban la tierra aprovechable. También, los maoríes de Nueva Zelanda se caracterizaron por la construcción de fortificaciones que aumentaban el prestigio de cada grupo en las luchas. Otro ejemplo es la Guerra de las Galias, un conflicto militar librado entre el procónsul romano Julio César y las tribus galas entre el año 58 a. C. y 51 a. C. En el curso de las mismas la República romana sometió a la Galia, extenso país que llegaba desde el Mediterráneo hasta el Canal de la Mancha. 

La Guerra de los Cien Años (Guerre de Cent Ans en francés, Hundred Years' War en inglés) fue un prolongado conflicto armado que duró en realidad 116 años (1337-1453) entre los reyes de Francia y los de Inglaterra. Esta guerra fue de raíz feudal, pues su propósito no era otro que dirimir quién controlaría las enormes posesiones de los monarcas ingleses en territorios franceses desde 1154. Más recientes, se tienen los ejemplos de las Guerras Mundiales y la Guerra Fría en el siglo pasado. Con el paso del tiempo, los hombres fueron evolucionando, y así también lo hicieron los objetivos de los conflictos bélicos, los armamentos y estrategias utilizados en estos conflictos.

El objetivo de este proyecto es el desarrollo de un programa que permita la simulación de diferentes batallas que se hayan producido en un pasado distante, en épocas más recientes e incluso simular batallas futuristas o con elementos de fantasía. Además, poder simular batallas entre diferentes épocas, por ejemplo enfrentar 20 soldados armados con AK-47 contra 1000 soldados armados con espadas y escudos, en un terreno montañoso.

El proyecto consiste en dos grandes módulos:

\begin{itemize}
    \item \textbf{Core:} comprende todo lo relacionado con la simulación de los enfrentamientos bélicos: la definición y generación de mapas y terrenos, el funcionamiento de las unidades básicas y el funcionamiento del simulador.
    \item \textbf{Battle Script:} comprende la definición la gramática del lenguaje, y la implementación del compilador, entiéndase el tokenizador, el parser, la generación de código, entre otras funcionalidades.
\end{itemize}

